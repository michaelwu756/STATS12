\documentclass[12pt]{article}
\usepackage{amsmath}
\begin{document}
\title{Statistics 12, Homework 6}
\date{March 3rd, 2019}
\author{Michael Wu\\UID: 404751542}
\maketitle

\section*{Chapter 17, Problem 2}

\paragraph{a)}

I would expect \(51\%\) of the sample to be women.

\paragraph{b)}

\[\sqrt{\frac{0.51\times0.49}{400}}\approx 0.024995 = 2.4995\%\]

\paragraph{c)}

I would expect \(0.51\times400=204\) women to be in the sample on average.

\section*{Chapter 17, Problem 4}

This demonstrates what the Central Limit Theorem says about the sampling distribution
for sample proportions because as the size of the sample increases, the distribution
becomes more normal. The center of the distribution remains at 0.85, which is the
given proportion. The shape is initially skewed left but becomes symmetric as
the sample size increases. The spread is initially wide, but becomes narrower as
the sample size increases. This fits with the Central Limit Theorem because it says the
standard deviation of the sampling distribution should scale inversely by a factor of
\(\sqrt{n}\), where \(n\) is the sample size.

\section*{Chapter 17, Problem 8}

\paragraph{a)}

I would be comfortable using the normal distribution with samples of size 100.

\paragraph{b)}

The success and failure condition says that this is an appropriate choice since at this
sample size the simulation has at least 10 success and failures.

\section*{Chapter 17, Problem 10}

\paragraph{a)}

No, a normal model would not be a good model for this sampling distribution because it will
always be skewed right. The squared term in the variance means that cans that are far away
from the 16 oz target affect the variance more than cans that have close to 16 oz of soup.

\paragraph{b)}

No I would not expect about 95\% of the samples to have their variances between 0.008 of
0.009. This is because the distribution is skewed right, so a larger percentage of the variances
would be outside this range on the right than they should be.

\section*{Chapter 17, Problem 12}

\paragraph{a)}

\[\sqrt{\frac{0.76\times0.24}{500}}\approx 0.0191 = 1.91\%\]

\paragraph{b)}

\[\sqrt{\frac{0.76\times0.24}{125}}\approx 0.0382 = 3.82\%\]

\section*{Chapter 17, Problem 22}

The standard deviation of the sample proportion should be
\[\sqrt{\frac{0.1\times0.9}{500}}\approx 0.01341 = 1.341\%\]
and so the z-score of the observed percentage of M\&M's is
\[\frac{0.12-0.1}{0.01341}\approx 1.49\]
This is not an unusual z-score since it is within two standard deviations from
the given proportion.

\section*{Chapter 18, Problem 4}

\paragraph{a)}

The interval means that from this sample, we can predict that the true population proportion
of people who say that the trend of more single women having children is a good thing should lie
between 2.9\% and 5.1\% with 95\% confidence.

\paragraph{b)}

The meaning of 95\% confidence is that our generated interval should contain the true population
proportion 95\% of the times we perform this study using the same methods.

\section*{Chapter 18, Problem 6}

\paragraph{a)}

The critical value for a 99\% confidence level is 2.58. Then the following calculation gives us
the margin of error.
\[2.58\times\sqrt{\frac{0.49\times0.51}{1500}}\approx0.0333=3.33\%\]

\paragraph{b)}

The margin of error should be smaller for 95\% confidence. This is because the critical value required
for 95\% confidence should be smaller, while the standard error remains the same.

\section*{Chapter 18, Problem 8}

\paragraph{a)}

Yes, the conditions for constructing a confidence interval are met. The respondents are randomly
selected and independent, the number of successes and failures are above at least 10, and the sample size
is less than 10\% of the population size.

\paragraph{b)}

\begin{align*}
    1.96\times\sqrt{\frac{0.04\times0.96}{n}}&=0.03\\
    \frac{0.04\times0.96}{n}&=\left(\frac{0.03}{1.96}\right)^2\\
    n&=0.04\times0.96\times\left(\frac{1.96}{0.03}\right)^2=163.9
\end{align*}
At least 164 people should be surveyed to create a 95\% confidence interval with a margin of error less than
3\%.

\section*{Chapter 18, Problem 14}

\paragraph{a)}

This conclusion is incorrect. We do not know how to define fairness and our study does not even mention the
concept.

\paragraph{b)}

This conclusion, while technically correct, is inaccurate because we know the true experimental proportion
of times the euro landed on heads. It would be more correct to say that in this experiment, the euro landed
on heads 56\% of the time.

\paragraph{c)}

This conclusion is correct. The confidence interval indicates that the true proportion that a euro
will land on heads lies within 51\% and 61\% with 90\% confidence.

\paragraph{d)}

This conclusion is correct. Assuming that we are able to observe enough spins such that the standard deviation
of the sample proportion becomes very small, the proportion of observed heads becomes very close to the
true percentage that a euro will land on heads. We are 90\% confident that the true proportion that a euro
will land on heads on heads lies within 51\% and 61\%.

\paragraph{e)}

This conclusion is correct. It is exactly the same as the previous conclusion.

\end{document}