\documentclass[12pt]{article}
\usepackage{amsmath}
\begin{document}
\title{Statistics 12, Homework 4}
\date{February 17th, 2019}
\author{Michael Wu\\UID: 404751542}
\maketitle

\section*{Chapter 11, Problem 6}

Yes, this is a simple random sample because each employee has the same probability of being
chosen and each number is assigned randomly.

\section*{Chapter 11, Problem 20}

\paragraph{a)}

The population is all the soil at locations near the former industrial waste dump.

\paragraph{b)}

The population parameter of interest is the true levels of toxic chemicals in the
soil.

\paragraph{c)}

The sampling frame is the soil at locations near the former industrial waste dump.
We do not know exactly how the researchers defined this, perhaps they used
the distance from the site of the waste dump or some other method do define the
region in which they took samples.

\paragraph{d)}

The sample is the 16 peices of soil that were collected.

\paragraph{e)}

The sampling method in this study is unknown. We do not know if randomization
was employed. Perhaps the soil locations were determined by hand to ensure good
coverage or perhaps they were generated by randomly selecting points on a map with
a computer.

\paragraph{f)}

Soil at locations far away from the former waste dump were left out of the study.
This is presumably because the EPA thinks that the waste dump would not have
an effect on soil at locations far away from the former waste dump.

\paragraph{g)}

Potential problems could result from the unknown sampling method and how the researchers
defined the sampling frame. If the researchers made the sampling frame cover too large of an
area or sampled at locations that were shielded from toxic chemicals that came from the waste
dump, they may have incorrectly concluded that there were no harmful substances.

\section*{Chapter 11, Problem 24}

The station's faulty prediction is more likely to be a result of bias.
The sample was voluntary and consisted of viewers of the TV station. So the set of
responders to the poll could be biased and not reflect the true population of voters.
I assume that many people responded, so if there was no bias the sampling error
should be small since it decreases with a larger sample.

\section*{Chapter 12, Problem 14}

Confounding factors could include the sunlight, water, and ambient temperature
of the environment that the tomatoes grow in. If the study does not control
these, they may see inconsistent results.

\section*{Chapter 12, Problem 16}

\paragraph{a)}

This is not an experiment. It is an observational study since the researcher simply
observed data that was gathered in the past.

\paragraph{b)}

No, we cannot conclude that shorter men are at higher risk of dying from a heart attack.
This is because observational studies may not take into account confounding variables.
Perhaps the shorter men had poor nutrition or came from a poorer area that caused
both the shortness and risk of heart attack.

\section*{Chapter 13, Problem 6}

\paragraph{a)}

The probability that a randomly played song is a rap song is 35\%.

\paragraph{b)}

The probability that a randomly played song is not country is 85\%.

\section*{Chapter 13, Problem 8}

\paragraph{a)}

The probability that all these adults have only a cell phone and no
landline is \(0.49^4=5.764801\%\).

\paragraph{b)}

The probability that none of these adults have only a cell phone
and no landline is \(0.51^4=6.765201\%\).

\paragraph{c)}

The probability that at least one of these adults has only a cell
phone and no landline is \(1-0.51^4=93.234799\%\).

\section*{Chapter 13, Problem 10}

\paragraph{a)}

No the chances of different sums have different probabilities.
There is one way to make a sum of two, two ways to make a sum of three,
three ways to make a sum of four, four ways to make a sum of five, five
ways to make a sum of six, six ways to make a sum of seven, five ways to
make a sum of eight, four ways to make a sum of nine, three ways to make a sum
of ten, two ways to make a sum of eleven, and one way to make a sum of twelve.
The following table shows the sample space and probabilities.
\[
    \begin{array}{c|c}
        \text{Sum} & \text{Probability} \\
        \hline
        2 & \frac{1}{36} \\
        3 & \frac{2}{36} \\
        4 & \frac{3}{36} \\
        5 & \frac{4}{36} \\
        6 & \frac{5}{36} \\
        7 & \frac{6}{36} \\
        8 & \frac{5}{36} \\
        9 & \frac{4}{36} \\
        10 & \frac{3}{36} \\
        11 & \frac{2}{36} \\
        12 & \frac{1}{36}
    \end{array}
\]

\paragraph{b)}

The events are equally likely in this sample space, assuming the probability of having
a male or female is always \(\frac{1}{2}\). The sample space contains the following
eight events, each of which have a probability of \(\frac{1}{8}\).
\[\{MMM, MMF, MFM, MFF, FMM, FMF, FFM, FFF\}\]

\paragraph{c)}

The events are not equally likely in this sample space. The following table shows
the sample space and probabilities, assuming the probability of getting a tail is
\(\frac{1}{2}\).
\[
    \begin{array}{c|c}
        \text{Number of Tails} & \text{Probability} \\
        \hline
        0 & \frac{1}{16} \\
        1 & \frac{4}{16} \\
        2 & \frac{6}{16} \\
        3 & \frac{4}{16} \\
        4 & \frac{1}{16}
    \end{array}
\]

\paragraph{d)}

The events are not equally likely in this sample space. There is only one way to
make the longest run of heads equal to 10, but there are multiple ways to
make it equal to 3. The sample space contains the following possible values
for the longest run.
\[\{0, 1, 2, 3, 4, 5, 6, 7, 8, 9, 10\}\]

\end{document}
