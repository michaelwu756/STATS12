\documentclass[12pt]{article}
\usepackage{amsmath}
\begin{document}
\title{Statistics 12, Homework 7}
\date{March 10th, 2019}
\author{Michael Wu\\UID: 404751542}
\maketitle

\section*{Chapter 19, Problem 6}

\paragraph{a)}

The null hypothesis is that the proportion of vacant houses in the county is equal to \(0.114\). The alternative
hypothesis is that the proportion of vacant houses in the county is not equal to \(0.114\).

\paragraph{b)}

We can verify that our sample is random, less than 10\% of the population, and has at least 10 successes and failures. So
we can use the normal model to perform a one proportion \(z\)-test.

\paragraph{c)}

We have the standard deviation of the sample proportion equal to \(\sqrt{\frac{0.114\times0.886}{850}}=0.0109\).
The sample proportion is \(\frac{129}{850}=0.1518\). The test statistic is then \(\frac{0.1518-0.16}{0.0109}=-0.7523\).
This has a p-value of \(0.4519\). If the null hypothesis is true, the sampling distribution of sample proportions will
look like the following distribution.
% TODO

\paragraph{d)}

Using a significance level of \(0.05\), we find that the p-value is greater than this and we cannot reject the null hypothesis.
Thus we cannot conclude that the proportion of empty houses in the county is different from the proportion of empty houses in the
United States.

\section*{Chapter 19, Problem 8}

Her alternative hypothesis should be two sided, since she is trying to find out if the proportion in the county is
different from the proportion in the nation as a whole. This ensures that she could detect both if the proportion in the
county is low enough or high enough to reject the null hypothesis.

\section*{Chapter 19, Problem 14}

Since the seller has not specified the proportion of times that a 6 should appear, we can only set up our hypothesis
test by using \(p=\frac{1}{6}\) as our null hypothesis and \(p>\frac{1}{6}\) as our alternative hypothesis. Let
us assume a significance level of \(0.05\). Then since our p-value is less than this, we can reject the null hypothesis
and conclude that the die is loaded. Thus we should choose choice d. 

\section*{Chapter 20, Problem 20}

\paragraph{a)}

This output means that we can say that the true average age that babies begin to crawl for our population is
between 29.202 weeks and 31.844 weeks with 95\% confidence.

\paragraph{b)}

\[\frac{31.844-29.202}{2}=1.321 \text{ weeks}\]

\paragraph{c)}

The margin of error would be smaller. This is because the lower confidence level means that our test would have a smaller
critical value, reducing the margin of error. This makes sense because a smaller margin of error increases the chance
that our confidence interval does not contain the true population mean, which is expected due to the lower confidence.

\section*{Chapter 20, Problem 24}

\paragraph{a)}

We must assume that the daily income that the garage generates is independent from any other day and randomly selected
from some nearly normal distribution. This is not too reasonable since garage revenues may go up or down depending on
whether events are happening around the garage or if it is shopping season or whatnot. We may still proceed with our tests,
keeping this in mind.

\paragraph{b)}

Our standard error is \(\frac{15}{\sqrt{44}}=2.261\). We have 43 degrees of freedom, so our critical value is 1.681. Thus
our confidence interval is the one shown below.
\[126\pm1.681\times2.261=(122.20,129.80)\]

\paragraph{c)}

This confidence interval means that we can say the true mean daily income for the parking garage is between \$122.20
and \$129.80 with 90\% confidence.

\paragraph{d)}

90\% confidence in this context means that assuming that the requirements for constructing a confidence interval are met,
we would expect that an interval constructed using our methods should contain the true mean daily income for the parking
garage 90\% of the time.

\paragraph{e)}

This is the same as the previous section.

\section*{Chapter 21, Problem 2}

\paragraph{a)}

The parameter is a proportion. The hypotheses are shown below.
\[H_0:p=0.65\qquadH_a:p\neq0.65\]

\paragraph{b)}

The parameter is a proportion. The hypotheses are shown below.
\[H_0:p=0.45\qquadH_a:p\neq0.45\]

\paragraph{c)}

The parameter is a proportion. The hypotheses are shown below.
\[H_0:p=0.067\qquadH_a:p\neq0.067\]

\paragraph{d)}

The parameter is a mean. The hypotheses are shown below.
\[H_0:\mu=17.3\qquadH_a:\mu\neq17.3\]

\section*{Chapter 21, Problem 4}

\paragraph{a)}

True.

\paragraph{b)}

False. A high p-value means that if the null hypothesis was true,
we would have a high likelihood of observing the sample that we took. It
does not favor the null hypothesis, it only indicates that we cannot reject
the null hypothesis.

\paragraph{c)}

False. We cannot conclude that the null hypothesis is true, we can only choose
to not reject the null hypothesis.

\paragraph{d)}

False. If the null hypothesis is true, you would expect to get a p-value of 0.01
or below about 1\% of the time.

\section*{Chapter 21, Problem 12}

\paragraph{a)}

A Type I error has been made.

\paragraph{b)}

No error has been made.

\paragraph{c)}

No error has been made.

\paragraph{d)}

A Type II error has been made.

\end{document}
